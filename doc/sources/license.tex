\documentclass[11pt]{article}
\usepackage{url}
\usepackage{times}
\usepackage[T1]{fontenc}
% Document typeset from the original document that was typeset by Jeremy Elson.
% This document typeset by Alex Fletcher <furry@circlemud.org> on Dec 9/2001

\addtolength{\topmargin}{-.5in}       % repairing LaTeX's huge  margins...
\addtolength{\textheight}{1in}        % more margin hacking
\addtolength{\textwidth}{1in}         % and here...
\addtolength{\oddsidemargin}{-0.5in}
\addtolength{\evensidemargin}{-0.5in}
\setlength{\parskip}{\baselineskip}
\setlength{\parindent}{0pt}

\title{CircleMUD License}
\author{Jeremy Elson}
\begin{document}

\maketitle

\begin{abstract}
This document presents the CircleMUD License and the DikuMud License.  Both of these licenses must be followed in order to legally use or modify CircleMUD or any part of it.
\end{abstract}

\section{CircleMUD License}
CircleMUD was created by:
\par
Jeremy Elson\newline
Department of Computer Science\newline
Johns Hopkins University\newline
Baltimore, MD  21218  USA\newline
\url{<jelson@circlemud.org>}
\par
CircleMUD is licensed software.  This file contains the text of the CircleMUD license.  If you wish to use the CircleMUD system in any way, or use any of its source code, you must read this license and are legally bound to comply with it.
\par
CircleMUD is a derivative work based on the DikuMud system written by Hans Henrik Staerfeldt, Katja Nyboe, Tom Madsen, Michael Seifert, and Sebastian Hammer.  DikuMud is also licensed software; you are legally bound to comply with the original DikuMud license as well as the CircleMUD license if you wish to use CircleMUD.
\par
Use of the CircleMUD code in any capacity implies that you have read, understood, and agreed to abide by the terms and conditions set down by this license and the DikuMud license.  If you use CircleMUD without complying with the license, you are breaking the law.
\par
Using CircleMUD legally is easy.  In short, the license requires three things:
\begin{enumerate}
\item You must not use CircleMUD to make money or be compensated in any way.
\item You must give the authors credit for their work.
\item You must comply with the DikuMud license.
\end{enumerate}

That's it -- those are the main conditions set down by this license. Unfortunately, past experience has shown that many people are not willing to follow the spirit of the license, so the remainder of this document will clearly define those conditions in an attempt to prevent people from circumventing them.

\subsection{You must not use CircleMUD to make money or be compensated in any way.}
The first condition says that you must not use CircleMUD to make money in any way or be otherwise compensated.  CircleMUD was developed in people's uncompensated spare time and was given to you free of charge, and you must not use it to make money.  CircleMUD must not in any way be used to facilitate your acceptance of fees, donations, or other compensation. Examples include, but are not limited to the following:
\begin{itemize}
\item If you run CircleMUD, you must not require any type of fee or donation in exchange for being able to play CircleMUD.  You must not solicit, offer or accept any kind of donation from your players in exchange for enhanced status in the game such as increased levels, character stats, gold, or equipment.
\item You must not solicit or accept money or other donations in exchange for running CircleMUD.  You must not accept money or other donations from your players for purposes such as hardware upgrades for running CircleMUD.
\item You must not sell CircleMUD.  You must not accept any type of fee in exchange for distributing or copying CircleMUD.
\item If you are a CircleMUD administrator, You must not accept any type of reimbursement for money spent out of pocket for running CircleMUD, i.e., for equipment expenses or fees incurred from service providers.
\end{itemize}

\subsection{You must give the authors credit for their work.}
The second part of the license states that you must give credit to the creators of CircleMUD.  A great deal of work went into the creation of CircleMUD, and it was given to you completely free of charge; claiming that you wrote the MUD yourself is a slap in the face to everyone who worked to bring you a high quality product while asking for nothing but credit for their work in return.
\par
Specifically, the following are required:
\begin{itemize}
\item The text in the `\texttt{credits}' file distributed with CircleMUD must be preserved.  You may add your own credits to the file, but the existing text must not be removed, abridged, truncated, or changed in any way. This file must be displayed when the `\texttt{credits}' command is used from within the MUD.
\item The ``\texttt{CIRCLEMUD}'' help entry must be maintained intact and unchanged, and displayed in its entirety when the `\texttt{help circlemud}' command is used.
\item The login sequence must contain the names of the DikuMud and CircleMUD creators.  The 'login sequence' is defined as the text seen by players between the time they connect to the MUD and when they start to play the game itself.
\item This license must be distributed AS IS with all copies or portions of the CircleMUD that you distribute, if any, including works derived from CircleMUD.
\item You must not remove, change, or modify any notices of copyright, licensing or authorship found in any CircleMUD source code files.
\item Claims that any of the above requirements are inapplicable to a particular MUD for reasons such as ``our MUD is totally rewritten'' or similar are completely invalid.  If you can write a MUD completely from scratch then you are encouraged to do so by all means, but use of any part of the CircleMUD or DikuMud source code requires that their respective licenses be followed, including the crediting requirements.
\end{itemize}

\subsection{You must comply with the DikuMud license.}   
The third part of the license simply states that you must comply with the DikuMud license.  This is required because CircleMUD is a DikuMud derivative. The DikuMud license is included below.
\par
You are allowed to use, modify and redistribute all CircleMUD source code and documentation as long as such use does not violate any of the rules set down by this license.
\par
--Jeremy Elson
\par
CircleMUD 3 -- Copyright \copyright 1994-2001, The CircleMUD Group\newline
Other portions copyright by authors as noted in ChangeLog and source code.

\section{DikuMud License}
Everything below this point is the original, unmodified DikuMud license. You must comply with the CircleMUD license above, as well as the DikuMud license below.

\begin{verbatim}
/* ************************************************************************
*  Copyright (C) 1990, 1991                                               *
*  All Rights Reserved                                                    *
************************************************************************* */

                             DikuMud License

                      Program & Concept created by


Sebastian Hammer
Prss. Maries Alle 15, 1
1908 Frb. C.
DENMARK
(email quinn@freja.diku.dk)

Michael Seifert
Nr. Soeg. 37C, 1, doer 3
1370 Copenhagen K.
DENMARK
(email seifert@freja.diku.dk)

Hans Henrik St{rfeldt
Langs} 19
3500 V{rl|se
DENMARK
(email bombman@freja.diku.dk)

Tom Madsen
R|de Mellemvej 94B, 64
2300 Copenhagen S.
DENMARK
(email noop@freja.diku.dk)

Katja Nyboe
Kildeg}rdsvej 2
2900 Hellerup
31 62 82 84
DENMARK
(email katz@freja.diku.dk)


This document contains the rules by which you can use, alter or publish
parts of DikuMud. DikuMud has been created by the above five listed persons
in their spare time, at DIKU (Computer Science Instutute at Copenhagen
University). You are legally bound to follow the rules described in this
document.

Rules:

   !! DikuMud is NOT Public Domain, shareware, careware or the like !!

   You may under no circumstances make profit on *ANY* part of DikuMud in
   any possible way. You may under no circumstances charge money for
   distributing any part of dikumud - this includes the usual $5 charge
   for "sending the disk" or "just for the disk" etc.
   By breaking these rules you violate the agreement between us and the
   University, and hence will be sued.

   You may not remove any copyright notices from any of the documents or
   sources given to you.

   This license must *always* be included "as is" if you copy or give
   away any part of DikuMud (which is to be done as described in this
   document).

   If you publish *any* part of dikumud, we as creators must appear in the
   article, and the article must be clearly copyrighted subject to this
   license. Before publishing you must first send us a message, by
   snail-mail or e-mail, and inform us what, where and when you are
   publishing (remember to include your address, name etc.)

   If you wish to setup a version of DikuMud on any computer system, you
   must send us a message , by snail-mail or e-mail, and inform us where
   and when you are running the game. (remember to include 
   your address, name etc.)


   Any running version of DikuMud must include our names in the login
   sequence. Furthermore the "credits" command shall always cointain
   our name, addresses, and a notice which states we have created DikuMud.

   You are allowed to alter DikuMud, source and documentation as long as
   you do not violate any of the above stated rules.


Regards,



The DikuMud Group

Note:

We hope you will enjoy DikuMud, and encourage you to send us any reports
on bugs (when you find 'it'). Remember that we are all using our spare
time to write and improve DikuMud, bugs, etc. - and changes will take their
time. We have so far put extremely many programming hours into this project.
If you make any major improvements on DikuMud we would be happy to
hear from you. As you will naturally honor the above rules, you will receive
new updates and improvements made to the game.

\end{verbatim}

\end{document}
\end
